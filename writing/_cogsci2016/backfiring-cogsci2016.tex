% 
% Annual Cognitive Science Conference
% Sample LaTeX Paper -- Proceedings Format
% 

\documentclass[10pt,letterpaper]{article}

\usepackage{cogsci}
\usepackage{pslatex}
\usepackage{apacite}
\usepackage{url}
\usepackage{graphicx}
\usepackage{caption}
\usepackage{subcaption}
\usepackage{listings}
\usepackage{color}
\usepackage{textcomp}
\usepackage{amsmath}
\usepackage{amssymb}
\usepackage{wrapfig}
\usepackage{lipsum}

\graphicspath{{figures/}}

\def\signed #1{{\leavevmode\unskip\nobreak\hfil\penalty50\hskip2em
  \hbox{}\nobreak\hfil(#1)%
  \parfillskip=0pt \finalhyphendemerits=0 \endgraf}}

\newsavebox\mybox
\newenvironment{aquote}[1]
  {\savebox\mybox{#1}\begin{quote}}
  {\signed{\usebox\mybox}\end{quote}}


\definecolor{Red}{RGB}{255,0,0}
\newcommand{\red}[1]{\textcolor{Red}{#1}}  


\title{It goes without saying: The pragmatics of backfiring utterances}

\author{{\large \bf Eleanor Chesnut*} (eleanor.chestnut@stanford.edu) and {\large \bf Michael Henry Tessler*} (mtessler@stanford.edu) \\
  Department of Psychology, Stanford University}


\begin{document}

\maketitle


\begin{abstract}
Here is an abstract.

\textbf{Keywords:} 
pragmatics; language; common ground; Bayesian model

\end{abstract}

\section{Introduction}

\section{A model of backfiring language}

Observing an event is often evidence for an underlying cause that could reliable give rise to the event.
For example, it is reasonable to fear a wet vacation when you disembark in your destination and find it raining. 
Similarly, observing a behavior can be taken as evidence for an underlying habit in a person. 
Seeing your friend's roommate doing her dishes provides some evidence that the roommate usually does her dishes. 
It is a more likely explanation than the alternative.

This raises an interesting issue for communication, however, because if an event always occurs, then----assuming the listener and speaker are in common ground about it---it is not informative to remark on it. 
A speaker who deliberately remarks ``My roommate did her dishes today.'' should believe that usually her roommate does not do her ishes, and believes his listener to believe this as well. 
A listener who does not actually know whether or not the speaker's roommate usually does her dishes will interpret the utterance as implicating that the roommate does not usually do her dishes.
We refer to the phenomena that a speech-act about an event can provide evidence against an underlying cause that would reliable give rise to the event as \emph{backfiring}. 

We formalize the backfiring inference as a probabilistic model of pragmatic communication where listener and speaker do not share common ground. 



\section{Experiment}

\subsection{Method}

\subsubsection{Participants}

\subsubsection{Materials}

\subsubsection{Procedure}

\subsection{Results}

\bibliographystyle{apacite}

\setlength{\bibleftmargin}{.125in}
\setlength{\bibindent}{-\bibleftmargin}

\bibliography{backfiring-cogsci2015}


\end{document}
